\documentclass{article}

\usepackage[margin=1in]{geometry}
\usepackage[table]{xcolor}
\usepackage{graphicx}
\usepackage{wrapfig}
\usepackage{titlesec}
\usepackage{titling}
\usepackage{setspace}

\titleformat{\section}
{\large\bfseries}
{Method \thesection}
{0em}
{: }[]

\renewcommand{\maketitle}{
 \hspace{-22px}  \textbf{
{\theauthor} \\
ASEN5158 \\
\today \\
{\thetitle}} \titlerule
}

\definecolor{lightgray}{gray}{0.8}

\begin{document}

  \title{Homework 2}
  \author{Connor Johnstone}

  \maketitle

  \vspace{20px}
  \textbf{
    Assignment:
    \begin{enumerate}
      \item Identify 3 methods, existing or proposed, to counteract radiation 
        exposure in space, can be physical or physiological, similar concepts 
        or completely different approaches
      \item Summarize the high level functionality of each process -- what it
        does
      \item Contrast notable benefits/challenges of each approach, think of 
        this as providing `trade study’ info, not a recommendation
    \end{enumerate}
  }

  \setstretch{1.5}

  \section{Passive Shielding with Water}

  The most simple form of radiation shielding is passive shielding;
  that is shielding that involves simply a physical barrier between the human 
  passengers and the harmful radiation sources. The simplicity and lack of power
  requirements for this method make it extremely attractive, however the amount
  of material generally required can be prohibitive from a mass/cost standpoint.

  So one method of mitigating this problem would be to use a material such as
  water. Water is a very effective radiation shield but also provides a whole
  host of other useful functions on the trip. The uses of water as a radiation
  shield vary. On one side, a crew could simply take up only as much water as is
  needed for the mission, but store it in strategic places to act as an addition
  barrier. On the other side, enough water could be brought up so as to complete
  shielf the crew from harmful radiation. A realistic mission scenario will
  probably find somewhere in between in which to operate, as enough additional
  water to effectively shield from radiation would be very heavy, and thus very
  expensive to launch all the way to wherever the crew is going.

  \section{Active Electrostatic Shielding}

  A slightly more complex approach would be an active shielding approach. The
  definition of what qualifies as ``active shielding'' vary widely, but in
  general this involves some form of electromagnetic barrier that would repel or
  deflect the charged radiation particles.

  The simplest of these approaches would be an electrostatic shield. This
  involves simply generating a powerful electric field in the habitable space
  for the crew, with the positive potential at the center and the negative
  potential facing the outside of the ship. This approach can be significantly
  lighter than the passive shielding approach, however, the trade space now has
  to include the power requirements of generating the field, as well as the
  dielectric breakdown strength of the materials that will need to be used. It
  can be quite challenging to produce such a powerful electric field in space.

  There are some other active shielding methods, such as magnetic shielding.
  However, although there are no known health risks associated with living in a
  powerful static electric field, there are known health risks to living in a
  powerful magnetic field. These would have to be accounted for. There are also
  some experimental approaches, such as plasma shielding, but these approaches
  significantly increase the complexity of the solution and don't have to same
  level of prior testing.

  \section{Antioxidant Regimes}

  A third, fairly unique, approach would be to attempt to mitigate the effects
  of the radiation in the human body, rather than the radiation itself. Since
  the issue with radiation is that it cause cell breakdown which can lead to
  cancer, some research has been made into antioxidant regimes. These
  antioxidants are known to prevent cell breakdown. However, the research has
  lately shown that they are only particularly effective at preventing a certain
  kind of cell death (apoptosis) and so they may not be widely effective in
  preventing the types of cancer that radiation exposure is likely to cause.
  However, a medical solution to the problem, should it be found, would prove to
  substantially reduce the difficulty in sustaining long-duration human
  spaceflight, so it is a worthy area of research.

\begin{thebibliography}{9}

  \bibitem{clark}
    Ashley Clark,
    \textit{Radiation Shielding Techniques for Human Spaceflight},
    Stanford University,
    2015.

  \bibitem{caffrey}
    J. P. McCaffrey, H. Shen, B. Downton, E. Mainegra-Hing,
    \textit{Radiation attenuation by lead and nonlead materials used in
    radiation shielding garments},
    Ottawa, Ontario,
    2007.

  \bibitem{frazier}
    Sarah Frazier,
    \textit{Real Martians: How to Protect Astronauts from Space Radiation on
    Mars},
    NASA Goddard Space Flight Center,
    2015.

  \bibitem{buhler}
    Charles R. Buhler,
    \textit{Analysis of a Lunar Base Electrostatic Radiation Shield Concept},
    Kennedy Space Center, Florida,
    2004.

\end{thebibliography}

\end{document}
